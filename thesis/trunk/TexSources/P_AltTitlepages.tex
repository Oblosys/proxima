% Simple title page for partial versions. Handy when a table of contents is included.

% \subtitle must be defined in Thesis.tex

\selectlanguage{english}
\newcommand{\engtitle}{Proxima:\\
a~presentation-oriented~editor\\
for~structured~documents}

\newcommand{\nltitle}{Proxima:\\
een~presentatiegerichte~editor\\
voor~gestructureerde~documenten}

\thispagestyle{empty}

\parbox{141mm}{
\begin{center}
  {\sffamily\bfseries\Huge\engtitle\\}
  \vspace{1cm}
  \selectlanguage{dutch}
  {\sffamily\Large\nltitle\\}
  \vspace{2cm}
  {\sffamily\Large BIJNA PROEFSCHRIFT\\}
  \vspace{2cm}
  {\sffamily\Large\subtitle}

  \vspace{0.2cm}
  \selectlanguage{english}
  {\sffamily\large (version: \today)\\}
  \vspace{2cm}
  {\sffamily\Large Martijn Michiel Schrage\\}
\end{center}
}

\clearpage
\thispagestyle{empty}
\marginpar{\parbox{147mm}{
\begin{tabular}[t]{ll}
promotores:
 & Prof.\,Dr.~S.\,D.~Swierstra\\ 
 & Prof.\,Dr.~J.~Jeuring\\
 & Prof.~L.\,G.\,T.\,M.~Meertens\\
% &  Instituut voor Informatica en Informatiekunde, Universiteit Utrecht
\\
\multicolumn{2}{l}{Instituut voor Informatica en Informatiekunde, Universiteit Utrecht}\\
\end{tabular}}}



%%% Local Variables: 
%%% mode: latex
%%% TeX-master: "thesis"
%%% End: 
