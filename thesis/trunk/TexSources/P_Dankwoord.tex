\selectlanguage{dutch}
\chapter*{Dankwoord}
\label{chap:dank}
\addcontentsline{toc}{chapter}{Dankwoord}

% no running headers for these small chapters?

%\vspace*{-2.0cm}\hfill\parbox{9cm}{{\em
%If you break open the cherry tree,
%Where are the flowers?\\
%But when spring time comes,
%See how they bloom!}\\
%\\
%Ikkyu, ca. 1400 \\
%}
%\vspace{1cm}

Al is het schrijven van een proefschrift een grotendeels solitaire bezigheid, je doet het toch zeker niet zonder hulp van een heleboel anderen. En dit is de uitgelezen plek om iedereen daar eens uitgebreid voor te bedanken.

Allereerst natuurlijk dank aan mijn promotoren, Doaitse Swierstra, Johan Jeuring en Lambert Meertens. Johan, als dagelijks begeleider heb je heel wat stukjes van mij langs zien komen, die je elke keer  weer van gedetailleerd commentaar voorzag. Ook kon ik, ondanks jouw steeds drukker wordende schema, altijd bij je terecht als ik iets wilde bespreken. Dat heeft me vaak weer een stapje in de juiste richting geholpen. Doaitse, ook met jou kon ik regelmatig over Proxima van gedachten wisselen. Bedankt voor je heldere en kritische blik op mijn idee\"en, ook al waren dat -- zoals je nog wel eens verzuchtte -- niet de idee\"en die jij oorspronkelijk voor mij in gedachten had.

\bc cynisme \ec 
% alinea voor Lambert? tegenstelling niet in Nl?
%Al was je een grootste deel van de tijd niet in Nederland, o
En ten slotte Lambert: onze langdurige brainstormsessies bij het koffiezetapparaat behoren tot de leukste herinneringen aan mijn promotieonderzoek. Een rode draad was er vrijwel nooit, en eventuele toehoorders zouden onze conclusies over het~al~dan~niet~in~chocola- tjes lopen misschien wat vreemd hebben gevonden, maar deze gesprekken hebben niettemin een zeer grote invloed gehad op de inhoud van dit proefschrift. Bedankt!


\selectlanguage{english}
I would also like to thank the members of the examination committee (Roland Backhouse, Paul Klint, Arno Siebes, and Masato Takeichi) for reading (and approving) the manuscript, and providing me with several useful comments.

And, continuing in English, a big thank you to Zhenjiang Hu, Shin-Cheng Mu, Masato Takeichi, and the other members of the Programmable Structured Documents group at the University of Tokyo. Your enthusiasm for Proxima, and the prospect of a visit to your research group in Japan after submitting the manuscript have been a wonderful motivation in the last stages of writing this thesis.
\selectlanguage{dutch}


In Centrumgebouw Noord werden de dagen altijd op een leuke manier onderbroken door de middag- en andere pauzes met mijn lunchmaatjes Arjan en Daan. Bedankt voor de gezelligheid en de vele interessante discussies en gesprekken, die soms zowaar over de informatica gingen. Ook bedankt alle kamergenoten die ik de afgelopen jaren in CGN voorbij heb zien komen, en in het bijzonder Silja. Leuk dat ik je na drie\"enhalf jaar nu eindelijk terug kan bedanken! Verder ben ik Ren\'e van Oostrum zeer erkentelijk voor het beschikbaar stellen van zijn \LaTeX-stijl, die de basis was voor de vormgeving van dit proefschrift.

Naast al deze informatici ben ik natuurlijk ook mijn familie en vrienden dankbaar. Het zijn er te veel om allemaal te vermelden, maar toch wil ik een aantal mensen even apart noemen. Ten eerste mijn paranimfen Pascal en Jan (Michiel). Jullie hebben het proces van het schrijven van dit proefschrift van dichtbij meegemaakt, en ook al weten jullie volgens mij (nog) niet veel over de inhoud, toch heb ik bij de totstandkoming ervan veel gehad aan jullie vriendschap.  Ook Eelke wil ik bedanken, met name voor een bijzonder zinvol gesprekje in het Lepelenpark, op een moment dat ik mijn promotietoekomst even wat minder helder voor ogen had. Bedankt ook Marcel en Mildred voor de gezellige etentjes, en Iris en Tessa, al heb ik jullie met alle drukte natuurlijk veel te weinig gezien de laatste tijd.

Verder zijn er nog veel mensen die me afleiding en steun bezorgd hebben. \bc en die mijn geklaag de laatste maanden getolereerd hebben.\ec Bedankt Ingrid, Miriam, Jan-Willem, Evelyne, Carlein, Joep, Cindy, Hester, en alle andere vrienden en familieleden die  hier niet genoemd staan. 

%De mensen die mij op nachtelijke uren met een glazige blik aankeken wanneer 
%ik vertelde dat ik onderzoek naar generieke editors deed. 

% iedereen die mijn geklaag heeft getolereerd en altijd maar zei dat het allemaal wel ging lukken.

% mensen op nachtelijke uren project uitleggen aan de hand van bierviltjes en asbakken, wat leidde tot fruitpers metafoor


% Joke doorzettingsvermogen
Tot slot is de laatste plek in dit dankwoord voor Joke. Je bent er weliswaar niet bij geweest, maar het doorzettingsvermogen dat ik soms moest aanspreken om dit boekje te schrijven is zonder twijfel van jou afkomstig. En wat zou het \bc toch \ec mooi geweest zijn als je het resultaat had kunnen zien.
%Het zou wel erg mooi geweest zijn als je het eindresultaat had kunnen zien.

\bigskip
Martijn
