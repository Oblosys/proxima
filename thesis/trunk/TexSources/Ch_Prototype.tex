\chapter{The Proxima Prototype}
% Reuse from Ch_EditModel

A layer in proxima
One layer
show edit diff etc. 
\section{Architecture}







\section{Renderer}
Why ObjectIO. 
Easy and fast.
Tcl is slow, problem with diffing. 
OpenGL: Alas, there are no real documents describing HOpenGL yet. Incomplete and under construction. Maybe some day ok. However, might be optimized for 3-d rendering of graphical objects instead of heaps of text with a few lines and symbols + some nice menus.
GTK+HS/GTK2HS No documentation yet. Windows GTK does not appear stable (GIMP may crash at any time). Because the development platform (== my computer) is a Windows machine right now, this is not a wise choice.
Prent C renderer by Xander is incomplete, hard to update and not very portable. Also socket communication is expensive.
For now platform dependence is not a problem.
Porting to different Renderer is relatively easy, because of Layered implementation.
Future Proxima will perhaps use a standard Haskell GUI, or dedicated Renderer.


Krasimir:

  * Gtk+HS
   * iHaskell
   * Gtk2HS
   * HTk
   * TclHaskell
   * Fudgets
   * FranTk
   * Object I/O
   * Yahu

None of them are capable for large scale development 
for some reason. The HTk, TclHaskell, FranTk and Yahu 
are based on Tcl/Tk. The Tcl/Tk backend are portable 
but slow. The Gtk+HS, iHaskell, Gtk2HS are based on
GTK. The backend is powerful and portable (GTK >= 2.0)
but I prefer to use just native libraries on 
different platforms. The Object I/O is Win32 specific.
I think that there are need of one portable and 
efficient library. It needs to have advantages of
existing library but must have single uniform
interface. I post this message to haskell@haskell.org 
because I think that development of nice libraries 
requires agreement of the entire Haskell Community on
basic design lines. I looking for peoples which are
interested in the development of "standard" GUI
library.
--

\section{Example Editors}
