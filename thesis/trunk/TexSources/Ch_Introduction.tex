\chapter{Introduction}
\label{chap:introduction}

\section{XML}
Basics of XML and DTD's  Schema. CSS. XSL
XML is EBNF 
Documents are trees


\section{Why proxima?}

complex, but:
new stuff:

-extra state
-also editing on all? levels

Proxima not about changing the world: struct editors, views, (THE?). editors have illogical/inconsistent features. Easiest to start all over with new model, however will frustrate users. Better to have a model that allow to express the familiar edit models.

\section{Outline of the Thesis}
* mention here or in chapter proxArch?:
Bootstrapping problem, descr. of levels layers and edit model. second is easier to give with in a formal story. however, design reasons for first are hard to comprehend without idea. So first informal arch and edit model and later the formal stuff

\begin{itemize}
\item 1 Editing XML and structured documents, and applications of a good editor
\item 2-3 Explain the Proxima Editor System
\item 4 Formal definition of the precise mappings in a proxima layer
\item 5 Formal description of a layered editor using invariants
\item 6-11 The proxima Layers 



\item 13 Formal description of the editor in Haskell
\item 14 The proxima prototype
\item More on the architecture combinators?

\end{itemize}