\subsection{Further development}
%          o perspectives for further development of this innovation and/or other technologies. 


When the web-based functionality has been added to Proxima, an interesting extension will be the connection of Proxima to a database. Documents are now internal data structures, but it will be interesting to obtain subtrees of the document from a database. Furthermore, More incrementality, so larger documents can be edited. Also focus on parsing. Now only graphs and text which is parsed. To do Yahoo pipes, more general graphical editing. Formula editing.

Also infrastructure, multiple requests, servers? 


An logical extension would be to establish connection to database. Instead of a document containing all values, the presentation will (also) contain content obtained from database. Possible.

Large documents. increases load on server. incremental behavior in parsing and presentation. small change on presentation, small change in doc small change in rendering updat.

More interesting edit model. Now graphs and text. drag and drop, connecting pipes. also incremental

\bl
\o Database connections
\o Incrementality in parser for larger documents
\o Graphical editing
\el

\section{Project setup}
%    * Project setup --organizational, technical, eventual partners, dependencies on other projects, licenses, and such;

Work can be done at UU. Software technology group. Swierstra.

Builds upon the Proxima research project.

Resulting code open source. Lot of expertise in the group.

\subsection{Planning}

%    * Project planning --milestones and related results;
The available time will be around 25 weeks

\bl
\o investigate Ajax an DHTML technologies
\o setup Web server on simple web server and implement simple Ajax client
\o determine format of rendering
\o implement web server connections, and url that describes document location.
\o perhaps make more efficient encoding of rendering.
\o add compression to data using gzip library
\o implement an editor plugin for wiki
\o server-side file handling
\o incrementality for block moves
\el

\subsection{Budget}
%    * Project budgeting;

\bc
personeelskosten
1425       70%
3176 bruto 70%  (+werkgeversaandeel)
4537       100%

waarschijnlijk + 500 aan overhead
geen btw

\ec

The requested 30.000 \euro will be used to employ one postDoc for a duration of approximately six months. The position will be  by the requestor, Martijn Schrage.

\subsection{Risks}
%    * Project risks --which risks can be overseen from the start of the project;

One risk is efficiency. Latency may cause unpleasant editing. But experiments are quite promising, and latency problems can be overcome by predictive rendering. 

Efficiency. Experiments are positive.

The biggest challenge is to create a community. Adoptation. community. 

\subsection{Dissemination of results}

Implement plugin for wiki

De aanvraag zegt:

\bl
\o Project results dissemination --how the project team is going to disseminate results and to whom, publicity, diffusion of the produced innovation;
\el


\subsection{Follow-ups}

%    * Possibly, follow-ups on the project. 

\bl
\o Mention NWO and incrementality stuff?
\o Connection with databases?
\el

