\documentclass[10pt]{article}

\usepackage{../Utils}
\usepackage{eurosym}
\bc

Web-based editing

vb google docs, wikipedia.

Web 2.0 requires editing. Examples wikipedia (more). Enter formula is just text. More interesting editors are desireable.






Proxima generic editor is system that can be used to create editors. Can be easily extended to run server-based with a light client.

Internet connections no longer slow and haperend. e-mail clients now have ajax interfaces. More small terminals (phones etc.) Time is ripe to make server based editors. Allowing advanced editing on these machines.


VNC shows that it works. Latency is a problem. will get better, and also Proxima can do much better than that.

minimale communicatie, minimale client belasting
  snelheid
  batterij







Introduction to generic editing, and Proxima's layered architecture. Proxima used to be server based.

**** Proxima

Description of the Proxima architecture and basic layers (take from paper)

Web based version is basically swapping the lower layer to render to command strings and changing the GUI to a HTTP server. Then add a light client to render and capture events.

The challenge however is rather than send entire screens over the net, to only send a minimal update on a key press or mouse event. We need incrementality. 

**** Incrementality

to make it faster, we need incremental behavior

**** Other kinds of editors

examples

**** Research questions

Summarizing, we can identify the following research questions.

Research questions:
  incrementality in presentation system. Incremental ag evaluation, incremental presentation.
  
  lot of work done, but we have a specific situation. maybe anotations(automatically inferred)
  
  incremental parsing
  
  related to that parsing two dimensional structures in incremental way.

  handling absence of network connection. 







More complicated editors. Pipes (Mashup?). Full visual parsing is tricky, but with incrementality, we can do better. Also parsing math, and parsing word-processor text (italic, bold, colors, etc.)















Lambert:
1 iets waar ze geld aan willen geven

  - hier zit de wereld op te wachten
  - tijd is rijp, computers snel genoeg
  - we hebben speelgoed ding
  
  
2 hoe gaan we het bereiken.

Web pagina aanpassen.

\ec



\title{}
\author{}
\date{}
\begin{document}
\maketitle


\section{Project Title, Acronym, Principal Investigator}
%% 1

\subsection{Project Title}
Multi-user editing in a structured document editor

\subsection{Project Acronym}
Proxima

\subsection{Principal Investigator}
Dr.~M.\,M.~Schrage

\subsection{Renewed Application}

This is not a renewed application.

%% 2
\section{Summaries}

\subsection{Summary}

\subsection{Samenvatting}


%% 3
\section{Classification: science field and discipline}

\noindent Computer Science:
NOAG-ict theme 3.7: Methods for design and development, main discipline
Programming Languages, and subsidiary disciplines Software Engineering and
Algorithms and Computation Theory.

%% 4
\section{Composition of the Research Group}

\begin{tabular}{l||l||p{0.7cm}||l||p{2.5cm}}
name & position & hours p/w & affiliation & expertise \\ \hline
dr.~M.\,M.~Schrage & ? & 32 & UU &  \\
prof.\,dr.~S.\,D.~Swierstra & ? & 1 &  UU & PL, CC, FP, TC \\
\end{tabular}\\
$\mbox{}$\\
where FP is Functional Programming, PL is Programming Languages, PA is Program Analysis,
CC is Compiler Construction, AL is algorithms, and TC is Tool Construction.

% 5
\section{Research School}
The proposed research will be carried out in the Software Technology
area of the IPA research school (Instituut voor Programmatuurkunde en Algoritmiek).

%% 6   Max. 3500 words for 6,7 & 8
\section*{6a~~Description of the Proposed Research}
\addtocounter{section}{1}


%wetenschappelijke vraagstelling en de beoogde onderzoeksresultaten;

%onderzoeksaanpak en �methodologie;

%wetenschappelijk belang en urgentie van het voorgestelde onderzoek;

%verhouding van het voorgestelde onderzoek tot overeenkomstig onderzoek dat elders verricht wordt;

%inpassing van het voorgestelde onderzoek in het lopende onderzoek van de groep(en) waarin de aan te trekken projectmedewerker werkzaam zal zijn.

Proxima~\cite{schrage04Proxima}.

\section*{6b~~Application perspective}

% 7
\section{Planning}

%%8
\section{Expected use of Instrumentation}

Not applicable.

%% 9
\section*{Literature}

\bibliographystyle{plain}
\bibliography{../proxima}


\noindent At most five references to work done by members of the research team, ordered chronologically:

\renewcommand\refname{The 5 key publications of the research team}

\begin{thebibliography}{10}

\bibitem{schrage04Proxima}
Martijn~M. Schrage.
\newblock {\em Proxima -- a presentation-oriented editor for structured
  documents}.
\newblock PhD thesis, Utrecht University, The Netherlands, Oct 2004.

\end{thebibliography}


%% 10
\section{Requested Budget}

We are applying for the total of one postdoc (1,0 fte) for three years.\\

\noindent(1) PhD student, postdoc or other personnel\\
\begin{tabular}{lll}
a) appointment   &   1,0 x \euro 174.911 & = \euro 174.911  \\
b) personal benchfee & 1,0 x \euro 5.000 & = \euro 5.000 \\
c) additional travelling budget & &  = \euro 0.0 \\
d) project related apparatus/software & & = \underline{\euro 0.0 ~~~~+}\\
Subtotal PhD student, postdoc or other personnel & & = \euro 179.911
\end{tabular}\\

\noindent(2) Investments

\noindent Not applicable.

\end{document}

